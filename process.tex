\chapter{Group Process}
\section{Methodology}
We altered our methodology on \date{16-02-2016}. We changed from the \ac{spa} to a system more resembling anarchy.
We kept the \ac{po} and \ac{ac} concepts in order to maintain some control, the product owner focuses on what is required to be in the document, the agile coach has a focus on facilitating and monitoring the process.

The reason for this was that we have found it harder to make a good 'team split' for this assignment. Additionally it seems that team members trust each other better to do their work, making the somewhat more free model more comfortable.

\section{Retrospective Previous Assignment}
Before we started on this assignment we have done a SCRUM-style retrospective on the previous assignment: listing `things we have done well' and `things we can do better'.
The items identified are listed listed in table~\ref{tab:ass1retro}.

\begin{table}[!ht]
    \centering
    \begin{tabular}{L{0,45\textwidth}L{0,45\textwidth}}
         \textbf{What did we do well?} & \textbf{What can we do better?}\\
         \midrule
            Free coffee                         & More work on Monday/Tuesday  \\ 
            Initiative \& commitment of group   & Consistency of references \\
            Topic Separation                    & Merging early / continuous integration  \\
            QA (specifically: grammar)          & Plan for document structure  \\
                                                & Structured meetings  \\
                                                & ``Always have a deliverable''  \\
                                                & Double reviewing  \\
                                                & Keeping Trello up to date. \\
    \end{tabular}
    \caption{Assignment 1 retrospective}
    \label{tab:ass1retro}
\end{table}

%Changed to share-latex.
%dont know if this still applies, as I've seen that you can check the changes. Reviewing is a bit harder though
%- Accountability vs. continuous integration trade-off 

\section{Tools}
For this assignment we use the same issue tracking tool as the previous one: Trello. It has worked well during the previous assignment and we expect that it will do so again, especially if team members implement the suggestion from the retrospective.

To fix our issue with reference consistency we switched to LaTeX for the document. LaTeX has a mature reference management system that is easy to use.
For the previous assignment Google docs was used. Luckily a similar platform exists for creating TeX files: ShareLaTeX\footnote{\url{www.sharelatex.com}}.
Like Google docs it allows for multiple people to collaborate on the same file, including an edit history.

Additionally, LaTeX facilitates early merging of the document. 
An issue with working on a large file on Google docs is that editors may get in each others way.
In LaTeX a document can be split up in multiple files that compile to one output file (e.g.: one file per chapter), allowing multiple people to work on the same document without them getting in each other's way.

\section{Roles}
The roles of the various team members are listed in table~\ref{tab:roles}.
The product owner and agile coach roles have been moved to allow different members to experience working in these roles.
In addition to that a separate Document Owner is assigned, requirement for this was that the person had to be experienced with LaTeX. 

\begin{table}[!ht]
    \centering
    \begin{tabular}{>{\bfseries}l l}
    Person                      & Roles\\
    \midrule
    Jelle van Assema            & Team Member                   \\ 
    Felix Barten                & Team Member                   \\
    Robert Diebels              & Team member, Product Owner                 \\   
    Jasper Dijt                 & Team Member, Agile Coach, Document Owner   \\
    Yoan-Alexander Grigorov     & Team Member                   \\
    Guido Loupias               & Team Member                   \\
    Edward Poot                 & Team Member                   \\
    Theologos Zacharopoulos     & Team Member                   \\
    \end{tabular}
    \caption{Role division for assignment 2}
    \label{tab:roles}
\end{table}

%\section{Task Creation}
%%Don't know if this is relevant so I'm just putting it in here
%%Not very I guess, kindof merged it with 'Task Division'.
%We decided to create tasks based solely on the deliver-ables which are expect.
%
%\textbf{Rationale:} This decision was taken in an effort to eliminate waste from the process.

\section{Task Division}
We decided to divide the tasks based on who ever would want to take them. This resembles Anarchy as we get to decide what we want to do. Possibly influences the end result.
Tasks are based on the basic layout of the document, in a `fill in the gaps' way.

A possible problem of this approach is Cherry-Picking by team members, the \ac{po} and \ac{ac} will monitor for this behaviour.

\section{Task Principles}
% These are my own ideas and I received little feedback on them. However I'll list them here.
These principle are based on several sessions of feedback we received from our lecturers.
Our lectures stressed that \emph{insights} and \emph{depth of knowledge} are the aspects of this course that contain its essence. We list the following principles for execution of tasks and use them where applicable:
\begin{compactenum}
  \item \textbf{Insights considering:}
  \begin{compactenum}
    \item What the organisation learned
    \item What we have learned
  \end{compactenum}
  \item \textbf{Depth:}
  \begin{compactenum}
    \item \textbf{Critical thinking:}
    \begin{compactdesc}
      \item[What:] Describe the problem context.
      \item[Why:] Describe why something solves the problem.
      \item[How:] Describe how the problem is solved.?
      \item[When:] Describe the  context in which the problem is solved.
    \end{compactdesc}
    \item \textbf{Balance:} Demonstrate that we have a balanced view of the subject.
  \end{compactenum}
\end{compactenum}