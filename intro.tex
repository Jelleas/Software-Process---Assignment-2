\chapter{Introduction}
In this document two methodologies are analysed, these are \ac{rup}, \ac{scrum} and one failed case, the Ariane 5. This analysis is done in four parts. 

Firstly, through means of the \ac{cmmi}. The result is an analysis of the methodologies through this model, which exposes strengths and weaknesses of the methodologies according to \ac{cmmi}. This analysis is concluded with a series of insights, that show where the methodologies fail, and how some of their shortcomings can be explained and improved upon. 

Secondly, \ac{rup} and \ac{scrum} are analysed from a Lean manufacturing process viewpoint. This section shows where and how the methodologies incorporate lean principles, and where they do not. The analysis follows up with argumentation on how both methodologies could argue they produce less waste. Simultaneously showing strong points of the methodologies and learning aspects within them.

Thirdly, two successful companies are analysed: Google as a software development company, and Ikea as a company from a different industry. This analysis is done through the \ac{7s} model, which shows their organisational approach, rather than just their methodology. Through the lessons learned in this analysis the applicability of \ac{scrum}, \ac{rup} and the methodology of the Ariane 5 are studied.

%How is power analysed?
Finally, the methodologies are analysed on their usage of power and control. That is, who in the methodology is given power and how this power is then controlled. The analysis of control is done from the viewpoint of four control systems: Diagnostic, Belief, Boundary and Interactive systems~\citep{simons1995control}. The RUP, SCRUM and Ariane 5 case will be assessed on these four control systems and the degree of power they offer the workers 

%TODO finish when Power is finished

