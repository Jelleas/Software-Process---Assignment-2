\chapter{Power}
This chapter assess power-relations and control mechanisms in \ac{rup}, Scrum and within \ac{esa}. We investigate control systems for each of these from the perspecitive of \citep{simons1995control}. We address \textit{diagnostic control}, \textit{beliefs systems}, \textit{boundary systems} and \textit{interactive systems}. Then, for each of the aforementioned methods and \ac{esa} we end with a section on power-relations.

\section{Empowerment in Processes}

\subsection{Scrum}

\subsubsection{Control Systems}
\begin{description}
\item[Diagnostic Control Systems]
Scrum uses \emph{product burn-down charts} and \emph{sprint burn-down charts} to track the progress of development not only for the current sprint but also for the whole project.
Using extrapolation, these charts allow managers and stake-holders to see whether or not the project is going to be delayed.

\item[Belief Systems]
Scrum values \emph{personal responsibility}. In order for \ac{scrum} to be successful it is expected from team members to show commitment and responsibility. Initially, personal responsibility is imposed upon the whole team during the Sprint Retrospective. During this meeting, the team discusses everything that can be improved in future. It is up to individuals personal responsibility to take actions and improve. In addition, Scrum Roles also push for personal responsibility during the development process. For example, the Scrum Master's personal responsibility is to inspect the Scrum implementation during the Sprints, and if the framework is \textit{broken}, Scrum Master's sole responsibility is to fix it \citep[page 6]{schwaber2011scrum}. 
Furthermore, the Product Owner has personal responsibility to maintain and protect the Product Backlog, which means that only he/she is responsible for Product Backlog's item priority \citep[page 5]{schwaber2011scrum}. 
Finally, we have the Development Team which has the sole responsibility of delivering finished Product items (e.g. marked as "Done"). The observed Increment can only evolve because of the Development team's activity \citep[page 5]{schwaber2011scrum}.

On the other hand, the practice of Commitment-Driven Sprint Planning implies that team participants take responsibility towards estimations and task commitments. Moreover, commitments are carefully discussed within the Scrum Team during the Sprint Planning. 

Because of its emphasis on personal responsibility, Scrum also values \emph{transparency}~\citep[page 4]{schwaber2011scrum}.
Without transparency, there is no way for the team to exercise its power because there is no leader.
Everybody has to be aware of the situation.

\item[Boundary Systems]
First, nobody should dictate how the Scrum Development Team organises itself.
This is an embodiment of the personal responsibility value.
Second, the product owner should not be a group of people but should always be a single person.
Third, the development team should not touch the product and sprint backlogs without the involvement of the product owner.
The sprint backlog is the personal responsibility of the product owner.
Finally, requirements should not change during a sprint.

\item[Interactive Control Systems]
First, there is the \emph{daily stand-up}.
During this process the team has a short meeting where everybody stands up, explains what they did the day before and what they will do that day.
This control exposes progress and keeps everybody informed.
Second, there is the \emph{sprint review}.
When a sprint finishes, the stakeholders and the Scrum Team get together to inspect what has been done.
This control is similar to the daily stand-up except that, because stakeholders are informed, a clearer picture of progress emerges.
Finally, there is the \emph{retrospective}.
When a sprint finishes, the team brainstorms about what was good and what was bad during the sprint.
This control exposes shortcomings during the process.

\end{description}

\subsubsection{Power}
A Scrum team is a very flat organisation in its core. Its empowerment side is focused mainly on individuals. The Scrum Guide \citep{schwaber2011scrum} defines the Development Team as a set of professionals. However, the definition of a \textit{professional} is unclear. 
There are no methods of empowering people, everybody has equal decision making power over the technical development, expect for the Scrum Master who is empowered to control the proper implementation of the Scrum Framework as a process.
Therefore, the pace of the Development Team is determined by the slowest person.
Unfortunately, this also implies that the most competent people are not empowered \citep[page 6]{schwaber2011scrum}:

\begin{quote}
Scrum recognises no titles for Development Team members other than Developer, regardless of the work being performed by the person; there are no exceptions to this rule;
\end{quote}

More specifically, Scrum is not descriptive enough about development practises that can empower experts during development. In contrast, other processes such as \acrfull{xp} define a set of rules about development practises with specific techniques and approaches \citep{www:www.extremeprogramming.org}. 

Because there is a risk that a significantly more competent person will be perceived as arrogant and be rejected by the rest of the team, such a person would need good leadership qualities to convince the rest of the team of his or her solutions \citep[page 116]{debuggingteams}.
In order to prevent these kinds of situations, Scrum does not provide us with answers.
So, the answer is to be found in other processes, for example, during the assembly of the team.

In Scrum the empowerment is also concentrated in the Development Team's self-organisation capabilities \citep[page 5]{schwaber2011scrum}:

\begin{quote}
Development Teams are structured and empowered by the organization to organize and manage their own work. The resulting synergy optimizes the Development Team’s overall efficiency and effectiveness.
\end{quote}

This shows that Scrum is aiming to empower the Development Team in terms of valuable delivery with every Sprint, similarly to the way Agile Manifesto imposes power of people over processes \citep{www:www.agilemanifesto.org}.

\subsection{RUP}
In this section a short analysis of the control systems present in \ac{rup} is done, followed by an analysis of the amount of power various persons involved with a \ac{rup} based project wield.

\subsubsection{Control Systems}
The different systems present will be categorised using the definitions from Robert Simons~\citep{simons1995control}: Diagnostic Control Systems, Interactive Control Systems, Boundary Systems and Beliefs Systems.

\begin{description}
\item[Diagnostic Control Systems] 
\Gls{rup} comes with a lot of elements that facilitate a diagnostic control system. 
These are the defined project phases with attached milestones consisting of goals and evaluation questions. Some of the artifacts produced, like the development plan and the risk list, also further facilitate this. Finally the assignment and tracking of issues, listed as an essential by Leslee Probasco~\citep{probasco2000ten}, completes the picture.
These elements combined can be used to track project process in a diagnostic manner, this tracking will in general be done by the person in the Project Manager role. This person will, most likely, also have to report back on this to the stakeholders.

\item[Belief Systems]
A beliefs system is present in every \ac{rup} based project team. It can be found in the key element Vision from Leslee Probasco~\citep{probasco2000ten}.
High level elements of this like the \emph{problem statement}, identification of the users and their needs define the mission of the project team, its reason to exist in the first place.
The vision is something that provides input at all levels of the development and communicates the fundamental why's and what's.

\item[Boundary Systems]
In a way \ac{rup} itself is a boundary system because it defines the minimum standards the project team will have to adhere to when it comes to the process to be followed.
The process provides the minimal standards of the tasks that have to be performed in a certain workflow.
Certain elements of the vision also constitute a boundary, such as the (non-)functional requirements and the design constraints. By providing both beliefs: the core mission of the team; and boundaries: minimal requirements, constraints; the vision provides an overview of the `sandbox' the team has to operate in.

%TODO: Find other boundary systems within rup // within produced artifacts // make clear this also includes the workflows, etc.
\item[Interactive Control Systems] 
Interactive Control is also present in \ac{rup}.
It is primarily found in the items Evaluation and Change Management identified by Leslee Probasco~\citep{probasco2000ten}.
After each development iteration the current product will be assessed.
As Probasco notes, the requirements will change during the project. This is likely to start happening as soon as the future users or stakeholders see the product for the first time.
Changes in requirements are strategic information pieces that are important for the success of the project, that require regular attention, and that are best discussed face to face.
In other words, the control system for changes used in the \ac{rup} is an Interactive Control System.
\end{description}

\subsubsection{Power}
In \ac{rup} there are two roles that act upon a control system. These are Project Management and Configuration and Control Management, these exert control by monitoring progress and/or potential changing requirements. Other elements that exert control, trough beliefs or boundaries, are the process itself and the stakeholders via their influence on the requirements.

Other workflows are only really in control with regards to their specific task.
e.g.: A person assigned to the test workflow has to verify the product based on the architecture and requirements. Someone assigned to the implementation workflow has only the code and unit tests to worry about, the rest is done in the Architecture and Design workflow, or the testing workflow.

So, assuming every person only has a task within a single \ac{rup} workflow, they will not have a lot of power.
And their sphere of influence is limited to their specific task.
Making the implementor, for example, not much more than a ``code monkey''.
However, it is very much possible to be assigned to multiple \ac{rup} workflows.
In smaller projects, say 4-5 developers, a project manager, a requirements engineer, and some testers, it is conceivable to have all developers also work on the architecture and design. Or just a subset of them that have proven to be good at that.

To wrap up, in \ac{rup} a person working in a single \ac{rup} workflow is only empowered to influence to project within his specific domain.
However, it is possible to empower competent individuals by having them work in more than one workflow.
This kind of empowerment may be limited by practical considerations. For example: a large project may need 20 developers, but not 20 architects.


\subsection{Ariane 5}
\subsubsection{Control Systems}

\begin{description}
\item[Diagnostic Control Systems] 
% Diagnostic Control 
As we have mentioned in the chapter on \ac{cmmi}, \ac{esa} places emphasis on \textit{what} should be done, not \textit{how} this should be done. This becomes blatantly clear when we read the foreword of \citep[3]{ecsQ80a}: "Requirements in this standard are defined in terms of what must be accomplished, rather than in terms of how to organise and perform the necessary work.". As with \citep{esaSEstandards1991} no specifications are given concerning project metrics.

However, this lack of specification does not mean there were no metrics. The standards state that: "The work schedule should show when the work packages are to be started and finished. A milestone chart shows key events in the project; these should be related to work package completion dates.". This could be seen as a measure of progress.

However, as this measure does not provide a \textit{how}, as with many of the suggested measures within project management, this might facilitate inadequate control \citep{simons1995control}. As managers are free to pick any measurement of their choosing. Managers are in fact empowered. The inadequacy comes into play because the people who are being managed are similarly empowered (they decided \textit{how} to implement technical details).

This mutual empowerment (in the \textit{how} sense) might cause problems, as managers would have incentive to implement measurements that backfire. 
This begs the question: "Does empowerment lead to entitlement?". If one empowered group forces control measures on another empowered group, do they do so because they feel entitled to such control?

Unfortunately we cannot answer these questions. Information regarding a well-founded insight into these questions is just not available for the Ariane case. However, we can confidently say that \ac{esa} does indeed mention the need for measurements within projects.

\item[Belief Systems]
If we look at the definition of beliefs systems from \citep{simons1995control} we see that beliefs systems are used to convey value and inspire employees. These are often short power-sentences such as: "Best customer service in the world!".

If we compare this to \ac{esa}'s purpose\footnote{\url{http://www.esa.int/About_Us/Welcome_to_ESA/ESA_s_Purpose}} we find a ginormous amount of text that is both dull and tedious to read.

From our point of view, this suggests that, structure-wise, combined with the fact that \ac{esa} focuses on a really technical goal, \ac{esa} is a technocratic organisation. The main reason is that, every internal process is exceptionally formal, without suggesting some form of creativity, specifically because the given the standards define explicitly ``what'' must happen.

As \citep{simons1995control} mentions, without a formal and concise beliefs system, employees often do not have a clear and consistent understanding of the values of the business and their place within the business. 

This in turn adds to a lack of creativity which means that employees are less likely to be motivated so search for new ways of adding value \citep[83]{simons1995control}. This stifles progress and improvement within an organisation and does not create opportunities for learning.

This forces us to consider the possibility that empowerment/freedom given to \textit{experts}, who decide \textit{how} implementation takes place, might lead to an environment with a lack of creativity and vision.

However, \ac{esa} does not "officially" list itself as a technocratic organisation. Even though it does show all the hallmarks, we cannot reliably conclude that \ac{esa} has no vision or decent beliefs system in place.

% Boundary systems
\item[Boundary Systems]
The boundary systems in Ariane 5 case is something that specifically addressed by the \ac{esa} process and development standards ~\citep{jones1997esa}, \citep{secretariat1996space}. The rules are strict in those documents although not in the sense that is presented in ~\citep{simons1995control}. \citep{simons1995control} presents the boundary systems as a set of rules that describe what to do, but not the opposite, what to ``not'' do. The reason is that by telling people what not to do allows innovation, but within clearly defined limits. However, the \ac{esa} standards define exactly what needs to be done, describing the process the deliverable the plan and everything else. Although, if you take the MRS (mars lander) standards~\citep{holzmann2006power}, those are a set of rules that define what \textit{not} to do, that may help the failed case not to fail in the end.

% Interactive Control Systems
\item[Interactive Control Systems]
As it is mentioned in~\citep{simons1995control} the interactive control systems enable the managers to observe uncertainties, threats and opportunities, and to respond proactively in case of problems. More specifically, for large organisations that the managers have less personal contact with people throughout the organisation, a system has to be defined in order to create some communication and information channels. In the case of \ac{esa} as it can be observed from their development standards~\citep{jones1997esa} and~\citep{secretariat1996space}, this is something that can be succeeded from the delivered documentation of every phase. \ac{esa} defines the software process on a waterfall-like format with different phases and items for each phase. However, the important issue here is how this process can provide enough information to the managers in order to track the strategic uncertainties that ``keep senior managers awake at night''~\citep{simons1995control}. We believe that this is something that could be used as an interactive control system since this process defines some explicit phases with \textit{hard milestones} between them with useful, for the control system, outcome. The deliverable of those milestones is a very detailed document that follows specific quality standards with exact numbers. Those quality standards must be conducted from a series of reviews and approval processes. The content of those document is something that can be easily used as \textit{Interactive Control Systems} in order to inform the exact state of the overall process with as much details as any manager want to dive into. This way a manager can get all the gathered information, interpret them for communication, which later can be used for future plans and actions.

\end{description}

\subsubsection{Power}
As is already mentioned above, \ac{esa} focuses on \textit{what} should be done and not \textit{how}. This is done explicit by stating that "Requirements in this standard are defined in terms of what must be accomplished, rather than in terms of how to organise and perform the necessary work."~\citep[3]{ecsQ80a}. This clearly suggests that the employees, and more specifically in this case, the \textit{experts} of \ac{esa} have the \textit{power} to make crucial decisions on the selection of the way of accomplishing the necessary work. 

This is something that is a result of the \textit{technocratic} approach of \ac{esa}'s development process, giving the power to the experts to make important decisions. However, that also suggests that the responsibility of those people is higher than the rest of the team since they are in charge of the most crucial decisions.

% \section{Lessons Learned}
%TODO: New sections.